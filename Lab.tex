%%%%%%%%%%%%%%%%%%%%%%%%%%%%%%%%%%%%%%%%%%%%%%%%%%%%%%%%%%%%%%%%%%%%%%
%%	Name: "Signal analysis template"
%%	File name: signalanalysis_template_main
%%	Version: 1.5
%%
%%	Compiler: XeLaTeX
%%
%%%%%%%%%%%%%%%%%%%%%%%%%%%%%%%%%%%%%%%%%%%%%%%%%%%%%%%%%%%%%%%%%%%%%%

\documentclass[conference,compsoc,onecolumn]{IEEEtran}

% *** LANGUAGE UTILITY PACKAGES ***
\usepackage[utf8]{inputenc} % Required for including letters with accents
\usepackage[spanish]{babel}

% *** USED PACKAGES ***
\include{usedpackages}

% *** CARPETA DONDE SE GUARDARAN LAS IMAGENES ***
\graphicspath{{figures/}}

% *** NUEVOS COMANDOS Y CONFIGURACIONES VARIAS ***
\interdisplaylinepenalty=2500
\newcommand{\Lpagenumber}{\ifdim\textwidth=\linewidth\else\bgroup
	\dimendef\margin=0
	\ifodd\value{page}\margin=\oddsidemargin
	\else\margin=\evensidemargin
	\fi
	\raisebox{\dimexpr -\topmargin-\headheight-\headsep-0.5\linewidth}[0pt][0pt]{%
		\rlap{\hspace{\dimexpr \margin+\textheight+\footskip}%
			\llap{\rotatebox{90}{\thepage}}}}%
	\egroup\fi}

\AddEverypageHook{\Lpagenumber}%

\newcommand{\newtxt}[1]{\textcolor{black}{#1}}
\renewcommand\IEEEkeywordsname{Palabras cláve:}
\newcommand{\mx}[1]{\mathbf{\bm{#1}}} % Matrix command
\newcommand{\vc}[1]{\mathbf{\bm{#1}}} % Vector command

%% Separación de palabras
\hyphenation{op-tical net-works semi-conduc-tor HHMMSS}


\begin{document}

% *** TITLES AND NAMES ***
% title of the document
\title{Laboratorio 01: Introducción a LATEX}
% author names and affiliations
\author{\IEEEauthorblockN{Laura Victoria García Moreno}
\IEEEauthorblockA{Codigo: 1022437085\\Escuela de Ciencias Exactas e Ingeniería\\
	Universidad Sergio Arboleda - Bogotá, Colombia\\
	laurav.garcia.correo@usa.edu.co}}


% *** MAKE TITLE ***
\maketitle
\IEEEoverridecommandlockouts
\IEEEpeerreviewmaketitle

\begin{abstract}
En el presente laboratorio se aplicarán las bases para desarrollar los posteriores laboratorios y talleres. Se presentarán las características de presentación de los laboratorios. Se realizará código de MATLAB que permita la generación, tratamiento y almacenamiento de imágenes. Se compilarán códigos en LATEXy se generarán archivos
PDF de acuerdo al formato de presentación.
\end{abstract}


\begin{IEEEkeywords}
    LATEX, PDF, Laboratorio, Análisis de señales.
\end{IEEEkeywords}


\section{Marco teórico}
\label{sec:introduction}
Escriba aquí su marco teórico.


\section{Resultados}
\label{sec:results}
% Escriba su texto aquí
La señal digital es un tipo de señal en que cada signo que codifica el contenido de la misma puede ser analizado
en término de algunas magnitudes que representan valores discretos, en lugar de valores dentro de un cierto rango.
Por ejemplo, el interruptor de la luz sólo puede tomar dos valores o estados: abierto o cerrado, o la misma lámpara:
encendida o apagada (véase circuito de conmutación). Esto no significa que la señal físicamente sea discreta ya que los
campos electromagnéticos suelen ser continuos, sino que en general existe una forma de discretizarla unívocamente.
Ver fig. 1. [[[[figura 1]

Los sistemas digitales, como por ejemplo el ordenador, usan la lógica de dos estados representados por dos
niveles de tensión eléctrica, uno alto, H y otro bajo, L (de High y Low, respectivamente, en inglés). Por abstracción,dichos estados se sustituyen por ceros y unos, lo que facilita la aplicación de la lógica y la aritmética binaria. Si el
nivel alto se representa por 1 y el bajo por 0, se habla de lógica positiva y en caso contrario de lógica negativa.

Los cuatro puntos cardinales principales son:

Norte

Sur

Este

Oeste

Los planetas del Sistema Solar, ordenados por proximidad al Sol, son:

1. Mercurio

2. Venus

3. Tierra

4. Marte…

xk+1 = xk + T vk cos(θk + ϕk + sk) + 1
2
T
2v˙k cos(θk + ϕk + sk). . .
. . . −
1
2
T
2vk
˙θk sin(θk + ϕk + sk)
yk+1 = yk + T vk sin(θk + ϕk + sk) + 1
2
T
2v˙k sin(θk + ϕk + sk). . .
. . . +
1
2
T
2vk
˙θk cos(θk + ϕk + sk)
θk+1 = θk + T ˙θk +
1
2
T
2 ¨θk
˙θk+1 = ˙θk + T ¨θk
vk+1 = vk + Tv˙k
ϕk+1 = ϕk + Tϕ˙
k
sk+1 = sk + Ts˙k
(1)
Código implementando MATLAB Recreamos los parámetros del gráfico de tipo tamaño de la caja y las rejillas,
etiquetas y leyendas


\section{Conclusiones}
\label{sec:conclusions}
% Escriba su texto aquí


\nocite{*}
\bibliographystyle{IEEEtran}
\label{sec:biblio}
% Descomente y modiffique el archivo biblio.bib para agregar bibliografía
%\bibliography{bib/biblio} 





%\pagestyle{empty}
\end{document}


