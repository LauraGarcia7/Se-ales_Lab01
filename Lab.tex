%%%%%%%%%%%%%%%%%%%%%%%%%%%%%%%%%%%%%%%%%%%%%%%%%%%%%%%%%%%%%%%%%%%%%%
%%	Name: "Signal analysis template"
%%	File name: signalanalysis_template_main
%%	Version: 1.5
%%
%%	Compiler: XeLaTeX
%%
%%%%%%%%%%%%%%%%%%%%%%%%%%%%%%%%%%%%%%%%%%%%%%%%%%%%%%%%%%%%%%%%%%%%%%

\documentclass[conference,compsoc,onecolumn]{IEEEtran}

% *** LANGUAGE UTILITY PACKAGES ***
\usepackage[utf8]{inputenc} % Required for including letters with accents
\usepackage[spanish]{babel}

% *** USED PACKAGES ***
% *** MISC UTILITY PACKAGES ***
\usepackage{comment}			% Agregar comentarios
\usepackage{lipsum}				% Inserts dummy text
\usepackage{blindtext}
\usepackage{listings}					% Coding
\usepackage{verbatim}				% Verbatim
\usepackage[final]{pdfpages}
\usepackage{booktabs,dcolumn}
\usepackage{pdflscape}
\usepackage{afterpage}
%\setlist[itemize]{noitemsep, nolistsep}
\usepackage[bookmarks=false]{hyperref}
\usepackage{tcolorbox}									% Coloured boxes, for LATEX examples and theorems, etc
\usepackage{color}
\usepackage{xcolor} % Required for specifying colors by name									% Color packages foreground and back­ground color man­age­men
% *** CITATION PACKAGES ***
\usepackage{cite}
% *** GRAPHICS RELATED PACKAGES ***
\usepackage{graphicx}
\usepackage{caption}
\usepackage{pgfplots}
\usepackage{tikz}
\usetikzlibrary{shapes,arrows}
\usetikzlibrary{decorations.pathmorphing} % noisy shapes
\usetikzlibrary{fit}					% fitting shapes to coordinates
\usetikzlibrary{backgrounds}	% drawing the background after the foreground
\pgfplotsset{compat=1.13}
% *** MATH PACKAGES ***
\usepackage{amsmath}
\usepackage{mathtools}
\usepackage{amssymb}
\usepackage{amsfonts}
\usepackage{expl3}
\usepackage{bm}

% *** SPECIALIZED LIST PACKAGES ***
\usepackage{algorithmic}
\usepackage{listings}					% Coding
\usepackage[framed,numbered,autolinebreaks,useliterate]{mcode}
% *** ALIGNMENT PACKAGES ***
\usepackage{array}
% *** SUBFIGURE PACKAGES ***
%\ifCLASSOPTIONcompsoc
%\usepackage[caption=false,font=normalsize,labelfont=sf,textfont=sf]{subfig}
%\else
%\usepackage[caption=false,font=footnotesize]{subfig}
%\fi
% *** FLOAT PACKAGES ***
\usepackage{fixltx2e}
\usepackage{stfloats}
%\fnbelowfloat
%\usepackage{dblfloatfix}
% *** PDF, URL AND HYPERLINK PACKAGES ***
\usepackage{url}
\usepackage{everypage}


\usepackage{multirow} % In order to be able to insert rows spanning multiple lines
\usepackage{verbatim}
\usepackage[all]{xy}
\usepackage{listings}
\usepackage{subfigure}
\usepackage{multibib}
\usepackage{setspace} 
\usepackage{algorithm}			    	  % To insert nice algorithms

% *** CARPETA DONDE SE GUARDARAN LAS IMAGENES ***
\graphicspath{{figures/}}
\usepackage{graphicx}

% *** NUEVOS COMANDOS Y CONFIGURACIONES VARIAS ***
\interdisplaylinepenalty=2500
\newcommand{\Lpagenumber}{\ifdim\textwidth=\linewidth\else\bgroup
	\dimendef\margin=0
	\ifodd\value{page}\margin=\oddsidemargin
	\else\margin=\evensidemargin
	\fi
	\raisebox{\dimexpr -\topmargin-\headheight-\headsep-0.5\linewidth}[0pt][0pt]{%
		\rlap{\hspace{\dimexpr \margin+\textheight+\footskip}%
			\llap{\rotatebox{90}{\thepage}}}}%
	\egroup\fi}

\AddEverypageHook{\Lpagenumber}%

\newcommand{\newtxt}[1]{\textcolor{black}}
\renewcommand\IEEEkeywordsname{\normalfont Palabras cláve:}
\newcommand{\mx}[1]{\mathbf{\bm{#1}}} % Matrix command
\newcommand{\vc}[1]{\mathbf{\bm{#1}}} % Vector command

%% Separación de palabras
\hyphenation{op-tical net-works semi-conduc-tor HHMMSS}


\begin{document}

% *** TITLES AND NAMES ***
% title of the document
\title{\normalfont Laboratorio 01: Introducción a \LaTeX{}}
% author names and affiliations
\author{\IEEEauthorblockN{Laura Victoria García Moreno}
\IEEEauthorblockA{Codigo: 1022437085\\Escuela de Ciencias Exactas e Ingeniería\\
	Universidad Sergio Arboleda - Bogotá, Colombia\\
	laurav.garcia@correo.usa.edu.co}}


% *** MAKE TITLE ***
 \maketitle
\IEEEoverridecommandlockouts
\IEEEpeerreviewmaketitle
\begin{abstract}

\normalfont En el presente laboratorio se aplicarán las bases para desarrollar los posteriores laboratorios y talleres. Se presentarán las características de presentación de los laboratorios. Se realizará código de MATLAB que permita la generación, tratamiento y almacenamiento de imágenes. Se compilarán códigos en \LaTeX{} y se generarán archivos
PDF de acuerdo al formato de presetación.
\end{abstract}


\begin{IEEEkeywords}
  \normalfont  LATEX, PDF, Laboratorio, Análisis de señales.
\end{IEEEkeywords}


\section{\normalfont Marco teórico}
\label{sec:introduction}
Escriba aquí su marco teórico.


\section{\normalfont Resultados}
\label{sec:results}
% Escriba su texto aquí
La señal digital es un tipo de señal en que cada signo que codifica el contenido de la misma puede ser analizado
en término de algunas magnitudes que representan valores discretos, en lugar de valores dentro de un cierto rango.
Por ejemplo, el interruptor de la luz sólo puede tomar dos valores o estados: abierto o cerrado, o la misma lámpara:
encendida o apagada (véase circuito de conmutación). Esto no significa que la señal físicamente sea discreta ya que los
campos electromagnéticos suelen ser continuos, sino que en general existe una forma de discretizarla unívocamente.
Ver fig. 1.
 \begin{figure}[h!]
\centering
\includegraphics[scale=0.5]{bib/.png}
\caption{Vehículo y su modelo cinemático.}
\label{fig:imagen laboratorio 1}
\end{figure}

Los sistemas digitales, como por ejemplo el ordenador, usan la lógica de dos estados representados por dos
niveles de tensión eléctrica, uno alto, H y otro bajo, L (de High y Low, respectivamente, en inglés). Por abstracción,dichos estados se sustituyen por ceros y unos, lo que facilita la aplicación de la lógica y la aritmética binaria. Si el
nivel alto se representa por 1 y el bajo por 0, se habla de lógica positiva y en caso contrario de lógica negativa.

Los cuatro puntos cardinales principales son:
\begin{itemize}
    \item Norte.
    \item Sur.
    \item Este.
    \item Oeste.
\end{itemize}

Los planetas del Sistema Solar, ordenados por proximidad al Sol, son:
\begin{enumerate}
    \item Mercurio
    \item Venus
    \item Tierra
    \item Marte...
\end{enumerate}
\begin{equation}\label{eq_ej}
\begin{split} 
x_{k+1}  = & x_{k}+ T_{v k} cos(\theta_{k} + \phi_{k} + s_{k}) +\frac{1}{2}T^{2}\dot{v}_{k}cos(\theta_{k} + \phi_{k} + s_{k})...\\
& ...-\frac{1}{2}T^{2}v_{k}\dot{\theta}_{k}sin(\theta_{k} + \phi_{k} + s_{k})\\
y_{k+1} = & y_{k}+ T_{v k}sin(\theta_{k} + \phi_{k} + s_{k}) +\frac{1}{2}T^{2}\dot{v}_{k}sin(\theta_{k} + \phi_{k} + s_{k})...\\  
&...+\frac{1}{2}T^{2}v_{k}\dot{\theta}_{k}cos(\theta_{k} + \phi_{k} + s_{k})\\
\theta_{k+1}= & \theta_{k} + T \dot{\theta}_{k}+ \frac{1}{2}T^{2}\ddot{\theta}_{k}\\
\dot{\theta}_{k+1}= & \dot{\theta}_{k} + T\ddot{\theta}_{k}\\
v_{k+1}= &v_{k} + T \dot{v}_{ k}\\ 
\phi_{k+1}= &\phi_{k} + T \dot{\phi} _{k}\\ 
s_{k+1}= & s_{k} + T\dot{s} {k}\\ 
\end{split}
\end{equation}


Código implementando MATLAB Recreamos los parámetros del gráfico de tipo tamaño de la caja y las rejillas,
etiquetas y leyendas

\lstset{language=Matlab, breaklines=true, basicstyle=\footnotesize}
\lstset{numbers=left, numberstyle=\tiny, stepnumber=1, numbersep=-2pt}
\begin{lstlisting}[frame=single]
  f_1 = 1.5e3; % [Hz]
  f_2 = 2*f_1; % [Hz]
  s1=cos(2*pi*f_1*t); %Carrier based on time vector; 
  s2=cos(2*pi*f_2*t); %Carrier based on time vector; 
  xmin=0; xmax=0.5; ymin=-1.1; ymax=1.1; %Box size 
  Box=[xmin xmax ymin ymax]; 
  labels = {'t, [ms] ' , 's_n(t) '}; %labelx , labely 
  legends = {'$s_1(t)$' , '$s_2(t)=-\int_0^\pi {s_1(t) dt}$'}; 
  h=legend(legends); 
  set(h, 'Interpreter ' , 'latex ' , 'Location ' , ... 
     'NorthEast' , 'FontSize ' ,16, 'FontWeight' , 'bold' , 'Orientation ' , 'vertical '); 
  xlabel(labels(1) , 'fontsize ' ,16, 'FontAngle' , 'Italic '); 
  ylabel(labels(2) , 'fontsize ' ,16); 
  xlim([Box(1) Box(2)]) ; ylim([Box(3) Box(4)]) ; 
  grid(gca, 'minor');
\end{lstlisting}


\nocite{*}
\bibliographystyle{IEEEtran}
\label{sec:biblio}
% Descomente y modiffique el archivo biblio.bib para agregar bibliografía
%\bibliography{bib/biblio} 





%\pagestyle{empty}
\end{document}


